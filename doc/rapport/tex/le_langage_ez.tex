\section{Le langage EZ}

Le langage EZ est un nouveau langage de programmation destiné aux débutants en
programmation.
L'esprit du langage est de limiter le plus possible les aspects techniques de
la programmation tout en offrant un langage performant, simple et expressif.

Le langage permet l'usage du paradigme de programmation impératif, et introduit
des notions de programmation fonctionelle.

Le langage supporte:
\begin{itemize}
    \item Programmation structurée (définition de fonctions, de procédures,
          de types de données définis par l'utilisateur, instruction de
          contrôle de flux).
    \item Récursivité (les fonctions peuvent s'appeller elles-même).
    \item Typage fort (les variables, paramètres de fonctions ont un type déterminé).
    \item Notions de "type d'accès" (on peut interdire la modifiction d'un
          paramètre par une fonction).
    \item Types de donnée complexe définis dans le langage comme les vectors
          (tableaux dynamiques) et les optionals
          (variables pouvant avoir ou non une valeur).
    \item Importation de fonctions utilisables en EZ et écrites en C++.
\end{itemize}

\subsection{Les éléments du langage EZ}

\subsubsection{Identifiants}
Les identifieurs sont utilisés pour nommer les différents éléments du langage,
comme les variables, les types de données (structures), les fonctions et les
procédure. Un indentifiant ne peut pas être un mot clé réservé du langage
(voir en annexe pour une telle liste).

\subsubsection{Les types primitifs du langage EZ}
Le langage EZ met à disposition un ensemble de types primitifs utilisables
à tout moment dans le programme.

\paragraph{Types simples (ou scalaires)}\mbox{} \\
Les types listés dans le tableau ci-dessous sont les types simples : \\
\begin{tabular}{ l | l | l | l }
    Nom de type     & Valeurs possibles      & Exemple de déclaration & Exemple de valeur \\
    boolean         & true, false            & local b is boolean     & b = true          \\
    integer         & valeurs d'un int64\_t  & local i is integer     & i = -42           \\
    natural         & valeurs d'un uint64\_t & local n is natural     & n = 42            \\
    real            & valeurs d'un double    & local r is real        & r = -0.42         \\
    char            & valeurs d'un char      & local c is char        & c = 'x'           \\
    string          & chaîne de caractère    & local s is string      & s = "toto"        \\
\end{tabular}

\paragraph{Le type vector}\mbox{} \\

Le type vector est plus complexe que les autres. Il permet de définir un tableau
dynamique d'éléments d'un type donné.

Par exemple, pour définir une variable locale de type "tableau dynamique d'entiers",
on écrit :
\begin{verbatim}
    local v is vector of integer
\end{verbatim}

On peut définir des vecteurs de vecteurs :
\begin{verbatim}
    local v is vector of vector of integer
\end{verbatim}

Pour accéder au nième élément d'un vector, la syntaxe est:
\begin{verbatim}
    v[n] = 42   // Assigne 42 au nieme element du vector v
\end{verbatim}

Lorsqu'une variable a le type vector, on peut utiliser les méthodes de vector
définis par le langage, comme dans l'exemple suivant:
\begin{verbatim}
    v.remove(n) // Retire le nième élément du vector v
    v.push(42)  // Ajoute 42 à la fin du vector v
\end{verbatim}

L'ensemble des méthodes de vector disponible est listé en annexe.


\paragraph{Le type optional}\mbox{} \\
Le type optional permet de définir des variables pouvant avoir une valeur ou
non. Cela est nécessaire pour définir des structures de données récursives
comme les arbres.

Pour définir une variable de type "entier optionel", on écrit:
\begin{verbatim}
    local x is optional integer
\end{verbatim}

Pour savoir si une variable de type optional a une valeur, on peut utiliser
la méthode \textbf{is\_set} :
\begin{verbatim}
    if x.is_set() then
        // Ici l'on peut accéder à la valeur de x sans risque
        print x.get(), "\n"
    endif
\end{verbatim}

Pour accéder à la valeur d'une variable optional, on utilise la méthode
\textbf{get} qui retourne une valeur du type de l'optional. Si la variable
optional n'est pas préalablement défini, le programme s'arrêtera.

Pour donner une valeur à une variable optional, on utilise la méthode
\textbf{set} :
\begin{verbatim}
    x.set(42)   // Maintenant, x.get() est l'entier 42
\end{verbatim}


\paragraph{Le type function}\mbox{} \\
Le type function permet de définir des variables qui sont en fait des fonctions.

Pour définir une variable de type fonction, on écrit :
\begin{verbatim}
    local f is function(in x is integer) return integer
\end{verbatim}

Pour assigner une valeur à une variable de type fonction, on utilise les lambda
functions (une partie de ce document sera dédié à expliquer de quoi il s'agit).
\begin{verbatim}
    f = lambda (in x is integer) return integer is return x * 2
\end{verbatim}


\subsubsection{Les valeurs}

Les valeurs sont l'un des éléments de base du langage EZ. On les rencontre
dans les expressions (voir ci-dessous).

Voici quelques exemples de valeurs :
\begin{verbatim}
    42          // Une valeur de type integer
    "toto"      // Une valeur de type string
    true        // Une valeur de type boolean
    x           // Une valeur référençant la valeur de la variable 'x'
    f(x)        // L'appel de la fonction 'f' avec pour paramètre la variable 'x'
    x.y         // Référence du membre 'y' de la structure 'x'
    empty Arbre // Optionelle de type Arbre (optional Arbre) vide
    v[32]       // Référence du 32eme membre du vecteur v
\end{verbatim}


\subsubsection{Les expressions}

Les expressions dans le langage EZ sont ce que l'on rencontre dans les
paramètres d'un appel de fonction, lors de l'affectation d'une variable, ou
bien lors de la définition d'une condition pour une instruction de type
contrôle de flux.

Les expressions peuvent être :
\begin{itemize}
    \item Des calculs arithmétiques ;
    \item Des comparaisons ;
    \item Des expressions booléennes ;
    \item Des appels de fonctions (ou procédures) ;
    \item La définition d'une fonction lambda ;
\end{itemize}

Voici quelques exemples d'expressions :
\begin{verbatim}
    2 + 3 * 4           // Calcul arithmétique
    x <= 2              // Comparaison
    x <= 2 and x >= 0   // Expression booléenne
    f(x)                // Appel de fonction

    /* Une définition de lambda function */
    lambda (in x is integer) return integer is return x * 2
\end{verbatim}

Certaines parties d'une expression peuvent être groupés en les plaçants
entre parenthèses :
\begin{verbatim}
    (2 + 3) * 4
    (2 + 5) * (x + 32)
\end{verbatim}

\paragraph{Calculs arithmétiques}\mbox{} \\

Le langage EZ donne accès à 5 opérateurs de calcul arithmétique :
\begin{verbatim}
    5 + 6       // Addition
    5 - 6       // Soustraction
    5 * 6       // Multiplication
    5.0 / 2     // Division
    5 % 3       // Modulo (ou reste de la division euclidienne)
\end{verbatim}

Pour l'opérateur '\%', les valeurs de part et d'autre de l'opérateur doivent
être des entiers.

\paragraph{Comparaisons}\mbox{} \\

Le langage EZ donne accès aux opérateurs de comparaisons suivants :
\begin{verbatim}
    x < y       // x est plus petit que y
    x <= y      // x est plus petit ou égale à y
    x > y       // x est plus grand que y
    x >= y      // x est plus grand ou égale à y
    x == y      // x est égale à y
    x != y      // x est différent de y
\end{verbatim}

\paragraph{Opérateurs booléens}\mbox{} \\

Le langage EZ donne accès à 3 opérateurs booléens :
\begin{verbatim}
    x and y     // x et y
    x or y      // x ou y
    not x       // non x
\end{verbatim}




\subsubsection{Les instructions du langage EZ}

Le langage EZ donne accès à différentes instructions :
\begin{itemize}
    \item L'instruction "print" ;
    \item L'instruction "read" ;
    \item L'instruction "return" ;
    \item Les affectations ;
    \item Les instructions de contrôle de flux ;
    \item Les appels de fonctions ;
\end{itemize}

\paragraph{L'instruction print}\mbox{} \\

Cette instruction permet l'affichage sur la sortie standard des valeurs
des expressions passés en paramètre.

Chaque expression donnée en paramètre à l'instruction print doit être séparé
par une virgule.

\begin{verbatim}
    // Affiche "Hello, world" et saute la ligne
    print "Hello, world\n"

    // Affiche "f(x) = " suivi de la valeur de l'appel de f(x) puis saute une
    // ligne
    print "f(x) = ", f(x), "\n"
\end{verbatim}


\paragraph{L'instruction read}\mbox{} \\

Cette instruction permet la saisie par l'utilisateur de la valeur d'une
variable.

Cette instruction ne prends qu'un unique paramètre qui doit être le nom d'une
variable.

\begin{verbatim}
    // Lit la valeur de la variable x
    read x
\end{verbatim}

La variable passé en paramètre doit avoir un type simple (entier, réel,
caractère, chaîne de caractères, booléen).


\paragraph{L'instruction return}\mbox{} \\

Cette instruction permet de définir la valeur de retour d'une fonction.

Cette instruction ne prends qu'un unique paramètre qui doit être une
expression.


\begin{verbatim}
    return 32
\end{verbatim}


\paragraph{Les affectations}\mbox{} \\

Les affectations permettent d'affecter à une variable une valeur.

\begin{verbatim}
    x = 42      // Affecte la valeur 42 à la variable x
    y = f(x)    // Affecte le résultat de f(x) à la variable y
\end{verbatim}


\paragraph{Les instructions de contrôle de flux}\mbox{} \\

Les instructions de contrôle de flux permettent d'exécuter un ensemble
d'instructions si la condition de l'instruction est remplie.

Ces instructions sont ou bien la traditionelle construction "if/elsif/else" ou
bien les boucles.

Voici les exemples de ces différentes instructions :
\begin{verbatim}
// Traditionnel if/elsif/else
if x < 10 then
    ndigits = 1
elsif x < 100 then
    ndigits = 2
else
    print "invalid number of digits"
    return 5
endif

// Sucre syntaxique qui équivaut à un "if" seul avec une unique instruction.
on x < 10 do print "x < 10 "

// Boucle while, affiche 10 fois "Je boucle".
while x < 10 do
    print "Je boucle\n"
    x = x + 1
endwhile

// Exécute les instructions entre "loop" et "until" jusqu'à ce que la condition
// soit vrai.
loop
    print "Je boucle\n"
    x = x + 1
until x > 10

// Exécute les instructions entre "do" et "endfor" pour chaque valeur de
// l'interval spécifié. Attention, la borne inférieur de l'interval est
// comprise, mais pas la borne supérieur (ici i prendra les valeurs de
// 0 à 9).
// La variable i doit avoir été préalablement définie.
for i in 0 .. 10 do
    print "Je boucle\n"
endfor
\end{verbatim}



\subsubsection{Les fonctions et procédures}

Un programme EZ est composé d'un ensemble de fonctions et procédures.

Une procédure est une fonction qui n'a pas de type de retour.

Une fonction a :
\begin{itemize}
    \item un nom (ou identifiant) ;
    \item un ensemble de paramètres ;
    \item un type de retour (pas pour les procédures) ;
    \item un ensemble de variables locales ;
    \item un ensemble d'instructions à exécuter (corps) ;
\end{itemize}

Voici quelques exemples de fonctions et de procédures :
\begin{verbatim}
function f(in x is integer) return integer
begin
    return x * 2
end

procedure lit_entier(out n is integer)
begin
    read n
end

function is_prime(in x is integer, in v is vector of integer) return boolean
    local i is integer
begin
    for i in 0 .. v.size() do
        on x % v[i] == 0 do return false
        on sqrt(x) < v[i] do return true
    endfor

    return true
end
\end{verbatim}

\paragraph{Les paramètres d'une fonction}\mbox{} \\

Les paramètres d'une fonction sont donnés entre les parenthèses suivants le
nom de la fonction. Un paramètre peut être utilisé ensuite dans le corps de
la fonction.

Un paramètre est composé de :
\begin{itemize}
    \item un type d'accès ;
    \item un nom ;
    \item un type ;
\end{itemize}

\begin{verbatim}
    <type d'accès> <nom> is <type>
\end{verbatim}

Le type d'accès détermine les possibilités d'accès du paramètre dans le corps
de la fonction. Ces types sont :
\begin{itemize}
    \item in : le paramètre peut uniquement être utilisé dans les expressions,
          sa modification est impossible ;
    \item out : la pramètre peut uniquement être modifié, l'accès à sa
          valeur est impossible ;
    \item inout : le paramètre peut à la fois être modifié et lu.
\end{itemize}


\paragraph{Définition de variables locales}\mbox{} \\

Après la ligne définissant la signature de la fonction, on peut rajouter un
ensemble de variables locales (une par ligne). Ces variables sont ensuite
utilisables dans le corps de la fonction.

Pour définir une variable locale, on utilise la syntaxe :
\begin{verbatim}
    local <nom> is <type>
\end{verbatim}

Quelques exemples :
\begin{verbatim}
    local x is integer  // Défini une variable locale x de type integer
    local v is vector of integer // Défini une variable locale x de type vecteur d'entiers.
\end{verbatim}

\paragraph{Le corps de la fonction}\mbox{} \\

Les instructions définissant le corps de la fonction sont placés entre les deux
mots clés "begin" et "end". "begin" est placé après la définition des
variables locales.


\subsubsection{Les structures de données}

Le langage EZ permet de définir ses propres types de données à l'aide
du mot clé "structure".

La syntaxe est :
\begin{verbatim}
structure <nom> is
    <liste de membres>
end
\end{verbatim}

Un membre de structure est défini de cette manière :
\begin{verbatim}
    <nom> is <type>
\end{verbatim}

Voici un exemple d'utilisation :
\begin{verbatim}

structure Personne is
    nom is string
    prenom is string
end

procedure saisie_personne(out p is Personne)
begin
    print "Nom: "
    read p.nom
    print "Prénom: "
    read p.prenom
end

procedure afficher_personne(in p is Personne)
begin
    print p.prenom, " ", p.nom
end

\end{verbatim}


\subsubsection{Les constantes}

Les constantes sont des valeurs immuables définis globalement qui peuvent être
accédés dans toutes les epxressions du programme.

La syntaxe de définition d'une constante est :
\begin{verbatim}
constant <nom> is <type> = <expression>
\end{verbatim}

Exemple d'utilisation :
\begin{verbatim}
constant PI is real = 3.14

function aire_du_cercle(in rayon is real) return real
begin
    return r * PI * 2
end
\end{verbatim}


\subsubsection{Les variables globales}

Les variables globales sont des variables définis globalement qui peuvent être
accédés et affectés dans toute fonction du programme.

La syntaxe de définition d'une variable globale est :
\begin{verbatim}
constant STATUS_SUCCES is integer = 1
constant STATUS_ERREUR is integer = 2

global status is integer

procedure oups()
begin
    status = STATUS_ERREUR
end

procedure cool()
begin
    status = STATUS_SUCCES
end

procedure afficher_status()
begin
    if status == STATUS_ERREUR then
        print "oups"
    elsif status == STATUS_SUCCESS then
        print "cool"
    endif
end
\end{verbatim}


\subsubsection{Un programme en langage EZ}

On appellera les fonctions, procédures, structures, constantes et gloables
"entités" du programme.

Un programme doit commencer par le mot clé "program" suivi du nom du programme.
Ce nom détermine le nom de la fonction principale du programme.

On peut donner après cet en-tête un ensemble d'entités.

La fonction principale retourne un entier, et prends en paramètre un
vector de strings, correspondants aux paramètres donnés au programme lors
de son invocation.

Voici un exemple de programme complet :
\begin{verbatim}
program prime_numbers

function is_prime(in x is integer, in v is vector of integer) return boolean
    local i is integer
begin
    for i in 0 .. v.size() do
        on x % v[i] == 0 do return false
        on sqrt(x) < v[i] do return true
    endfor

    return true
end

function average(in primes is vector of integer) return real
    local i is integer
    local sum is real
begin
    sum = primes.reduce(lambda (in x is integer, in y is integer) return integer is return x + y, 0)
    return sum / primes.size()
end

function prime_numbers(in args is vector of string) return integer
    local primes is vector of integer
    local current_prime is integer
    local number_of_primes is integer
begin
    primes.push(2)
    current_prime = 3

    // Get the number of primes to fetch.
    if args.size() > 1 then
        number_of_primes = integer_from_string(args[1])
    else
        print "user defined"
        number_of_primes = 20
    endif

    while primes.size() < number_of_primes do
        on is_prime(current_prime, primes) do primes.push(current_prime)
        current_prime = current_prime + 2
    endwhile

    print "The first ", number_of_primes, " prime numbers are ", primes, "\n"
    print "Their average is ", average(primes), "\n"

    return 0
end
\end{verbatim}

\subsubsection{Ajouter des fonctions et types au langage EZ}

Le langage EZ peut être améliorer en spécifiant des fonctions et types
"builtins".

Ces fonctions sont listés dans le fichier "ez-builtins.ez". On liste dans
ce fichier la signature des fonctions et procédures builtins, et le nom des
types de données.

Par exemple :
\begin{verbatim}
builtin structure File
builtin procedure open_file(in path is string, in accessor is string, out file is File)
builtin function is_file_open(in file is File) return boolean
builtin procedure close_file(out file is File)
builtin function read_line_from_file(inout file is File) return string
builtin function is_file_over(in file is File) return boolean
\end{verbatim}

Pour définir ces fonction et types, on doit les écrires en C++ dans le
fichier "ez/functions.hpp" :
\begin{verbatim}
#ifndef _ez_functions_hpp_
#define _ez_functions_hpp_

#include <cstdlib>
#include <cstdio>
#include <ctime>
#include <cmath>

namespace ez {

typedef std::FILE* File;

void open_file(const std::string& path,
               const std::string& accessor,
               File& file)
{
    file = fopen(path.c_str(), accessor.c_str());
}

bool is_file_open(File file) {
    return file != NULL;
}

void close_file(File& file) {
    if (is_file_open(file)) {
        fclose(file);
        file = NULL;
    }
}

std::string read_line_from_file(File file) {
    static char buf[4096];
    fgets(buf, sizeof(buf) - 1, file);
    return std::string(buf);
}

bool is_file_over(File file) {
    return feof(file);
}

}

#endif
\end{verbatim}

On peut ensuite utiliser ces fonctions et types dans un programme EZ :
\begin{verbatim}
program cat

procedure display_file(in path is string)
    local file is File
    local line is string
begin
    open_file(path, "r", file)
    if (not is_file_open(file)) then
        print "couldn't open file ", path
    else
        while not is_file_over(file) do
            line = read_line_from_file(file)
            print line
        endwhile
        close_file(file)
    endif
end

function cat(in args is vector of string) return integer
    local i is integer
begin
    for i in 1 .. args.size() do
        display_file(args[i])
    endfor
    return 0
end
\end{verbatim}

