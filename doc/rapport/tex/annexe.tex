\section{Annexe}

\subsection{Grammaire du langage EZ}

 Les éléments contenus entre chevrons sont les tokens du langage :
\begin{verbatim}
<eol> -> Fin de ligne
\end{verbatim}

Le symbol '*' représente une règle de production nulle.

\begin{verbatim}
Commentaire -> // .* <eol> Commentaire
Commentaire -> /* .* */ Commentaire
Commentaire -> *

Identifier -> [a-zA-Z_][a-zA-Z0-9_]*

EnteteProgramme -> program Identifier <eol>

Type -> integer | real | string | vector of Type | Identifier

VariableGlobale -> Global Identifier is Type <eol>

Constant -> constant Identifier is Type = Value <eol>

Structure -> structure Identifier is <eol> Membres
Membres -> DefinitionVariable Membres
Membres -> end <eol>

DefinitionVariable -> Identifier is Type <eol>

TypeAcces -> in | out | inout

Parametres -> TypeAcces DefinitionVariable Parametres
Parametres -> , Parametres
Parametres -> *

Procedure -> procedure Identifier ( Parametres ) <eol> VariablesLocales Instructions end

Function -> function Identifier ( Parametres ) return Type <eol> VariablesLocales Instructions end

VariablesLocales -> *
VariablesLocales -> local Identifier is Type <eol> VariablesLocales

Instructions -> Instruction Instructions
Instructions -> *

Instruction -> ControleFlux
Instruction -> Affectation
Instruction -> Expression
Instruction -> Print
Instruction -> Read
Instruction -> Return

ControleFlux -> If | For | While | Loop | On

If -> if Expression then <eol> Instructions Elsif Else endif <eol>
Elsif -> elsif Expression then <eol> Instructions Elsif
Elsif -> *
Else -> else <eol> Instructions
Else -> *

Range -> Expression .. Expression
For -> for Identifier in Range do <eol> Instructions endfor <eol>

While -> while Expression do <eol> Instructions endwhile <eol>

Loop -> loop <eol> Instructions until Expression <eol>

On -> on Expression do Instruction <eol>

Affectation -> Identifier = Expression <eol>

Print -> print ListeExpression <eol>

Read -> read Varref <eol>

Return -> return Expression <eol>

Varref -> Identifier SuiteVarref
SuiteVarref -> . Varref
SuiteVarref -> ( ListeExpression ) SuiteVarref
SuiteVarref -> [ Expression ] SuiteVarref
SuiteVarref -> *

Valeur -> <integer> | <natural> | <real> | <caractère> | <chaine> | Varref

Signature -> ( Parameters ) return Type
Lambda -> lambda Signature is Instruction

Expression -> Valeur SuiteExpression
Expression -> not Expression
Expression -> ( Expression )
SuiteExpression -> <opérateur binaire> Expression

ListeExpression -> Expression ListeExpressionSuite
ListeExpression -> *
ListeExpressionPasVide -> Expression ListeExpressionSuite
ListeExpressionSuite -> , ListeExpressionPasVide
ListeExpressionSuite -> *

\end{verbatim}


\subsection{Méthodes du type vector}

Nous donnons en syntaxe EZ les différentes méthodes du type vector :

\begin{verbatim}

// Retourne le nieme élément du vecteur
function at(in n is natural) return Type

// Ajoute un élément en fin de vecteur
procedure push(in element is Type)

// Retire le dernier élément du vecteur
procedure pop()

// Insère un élément à un emplacement donné du vecteur
procedure insert(in n is natural, in v is Type)

// Retire un élément à un emplacement donné du vecteur
procedure remove(in n is natural)

// Retourne le nombre d'éléments du vecteur
function size() return natural

// Supprime tous les éléments du vecteur
procedure clear()

// Applique la fonction donnée à tous les éléments du vecteur
procedure map(in f is function(inout Type))

// Applique une réduction de gauche à droite sur le vecteur. Le paramètre
// 'initial_value' est l'élément initial donné à la réduction.
function reduce(in f is function(in Type, in Type)) return Type

// Retire tous les éléments à vrai par le prédicat donné
procedure filter(in predicate is function(in Type) return boolean)

\end{verbatim}

