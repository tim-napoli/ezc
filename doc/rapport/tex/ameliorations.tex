\section{Les pistes d'amélioration du langage}

\subsection{Les énumérations}

Le langage ne permet pas encore de définir d'énumérations. On peut cependant
lister des constantes.
L'avantage d'utiliser des énumérations est d'ordre syntaxique (simplicité
d'écriture) et sémantique (vérifier qu'une fonction prenant en paramètre une
valeur de type énumération est bien appellée avec une valeur de l'énumération).

\begin{verbatim}
enumeration Fruit is
    ANANAS,
    APPLE,
    PEACH,
    RASBERRY,
end

procedure print_fruit(in fruit is Fruit)
begin
    // ...
end

// Ok, value is member of Fruit :
print_fruit(ANANAS)

// Will fail, not member of fruit :
print_fruit(42)

\end{verbatim}


\subsection{Programmation modulaire}

Permettre d'écrire un programme dans plusieurs fichiers.
Par exemple, avoir un fichier de la sorte :

\begin{verbatim}
module Personne

structure Personne is
    nom is string
    prenim is string
end

function nouvelle_personne(in nom is string, prenom is string) return Personne
    local p is Personne
begin
    p.nom = nom
    p.prenom = prenom
    return p
end
\end{verbatim}

Dans un autre fichier

\begin{verbatim}
program Famille

use Personne

// On a compris...
\end{verbatim}

\subsection{Tuples}
Introduire le type "tuple of" permettant de donner des valeurs plus complexes :

\begin{verbatim}
function vector_add(in u is tuple of (real, real),
                    in v is tuple of (real, real))
        return tuple of (real, real)
begin
    return (u.first + v.first, u.second + v.second)
end
\end{verbatim}

\subsection{Enrichir le langage}

Profiter du mapping avec le langage C++ pour introduire davantage de builtins
(support OpenGL, les maps, les threads, accès au système de fichier, réseau...).

