\section{Implémentation du compilateur}

Cette section présente la réalisation du compilateur \textbf{ezc}.

Nous avons choisi le langage de programmation C99 pour réaliser ce projet.
Aucune autre bibliothèque que la bibliothèque standard n'a été utilisé. Le
projet est donc à priori compatible POSIX.

\subsection{L'analyseur}

La technologie d'analyse utilisée est inspiré de la bibliothèque
Parsec\footnote{\url{https://wiki.haskell.org/Parsec}} d'Haskell.

Ce type d'analyseur est un analyseur fonctionnel, c'est à dire qui utilise des
fonctions pour analyser le texte.

L'analyseur résultant est un analyseur fonctionnant en récursivité à gauche,
c'est à dire que l'on analyse l'entrée de gauche à droite (analyseur LL(n).
De plus, cet analyseur fonctionne de haut en bas, c'est à dire que l'on débute
l'analyse par le plus haut niveau de l'arbre d'analyse.

Voici les 5 fonctions de base utilisées par tout le reste du parser EZ :
\begin{verbatim}
typedef enum {
    PARSER_SUCCESS,
    PARSER_FAILURE,
    PARSER_FATAL,
} parser_status_t;

parser_status_t char_parser(FILE* input, const char* allowed, char** output);

parser_status_t chars_parser(FILE* input, const char* allowed, char** output);

parser_status_t word_parser(FILE* input, const char* word, char** output);

parser_status_t until_char_parser(FILE* input, const char* c, char** output);

parser_status_t until_word_parser(FILE* input, const char* word, char** output);
\end{verbatim}
Chaque fonction a un prototype similaire, et retourne un "status de parseur" :
\begin{itemize}
    \item PARSER\_SUCCESS: la fonction a réussi à analyser l'entrée donnée ;
    \item PARSER\_FAILURE: la fontion n'a pas réussi à analyser l'entrée donnée ;
    \item PARSER\_FATAL: la fonction a échoué et le parsing doit s'arrêter ;
\end{itemize}

On utilise ensuite des fonctions d'analyse que l'on appelle "de combinaison"
pour combiner les fonctions d'analyse de base afin de créer des fonctions plus
complexes.

Afin de nous assister dans la réalisation de ces fonctions, nous avons rédigé
un ensemble de macro C :
\begin{verbatim}
/* Parse using a parser func. */
#define PARSE(_parser) \
    { \
        long _status = (_parser); \
        if (_status == PARSER_FAILURE) { \
            return PARSER_FAILURE; \
        } else if (_status == PARSER_FATAL) { \
            return PARSER_FATAL; \
        } \
    }

/* Parse using a parser func. If the function failed, return PARSER_FATAL and
 * print the given error message.
 */
#define PARSE_ERR(_parser, _err_msg) \
    if ((_parser) == PARSER_FAILURE) { \
        int _line, _column; \
        char _c; \
        /* TODO input in PARSE_ERR args */ \
        get_file_coordinates(input, &_line, &_column, &_c); \
        fprintf(stderr, "error (line %d column %d at '%c'): " _err_msg "\n", \
                _line, _column, _c); \
        return PARSER_FATAL; \
    }

/* Try to parse using a parser func. If the function failed, reset the
 * offset of `input` stream to its original seek. Return the parser status.
 */
#define TRY(_input, _parser) \
    ({ \
        long _offset = ftell(_input); \
        parser_status_t _try_status = (_parser); \
        if (_try_status == PARSER_FAILURE) { \
            fseek(_input, _offset, SEEK_SET); \
        } else if (_try_status == PARSER_FATAL) { \
            return PARSER_FATAL; \
        } \
        _try_status; \
    })

/* Try the parser until it return PARSER_FAILURE. */
#define SKIP_MANY(_input, _parser) \
    { \
        while ((TRY(_input, _parser)) != PARSER_FAILURE) { } \
    }
\end{verbatim}

A titre d'illustration, voici le code de l'analyse d'un ensemble d'expressions
séparés par une virgule, comme ce que l'on rencontre lors de l'appel d'une
fonction :
\begin{verbatim}
parser_status_t parameters_parser(FILE* input, context_t* ctx,
                                  parameters_t** parameters)
{
    expression_t* expr = NULL;

    if (TRY(input, expression_parser(input, ctx, &expr)) == PARSER_SUCCESS) {
        parameters_add(*parameters, expr);

        SKIP_MANY(input, space_parser(input, NULL, NULL));

        if (TRY(input, char_parser(input, ",", NULL)) == PARSER_SUCCESS) {
            SKIP_MANY(input, space_parser(input, NULL, NULL));

            PARSE_ERR(parameters_parser(input, ctx, parameters),
                      "invalid parameter");
        }
    }

    return PARSER_SUCCESS;
}
\end{verbatim}
La grammaire de cet analyseur pourrait être :
\begin{verbatim}
    PARMATERS -> *      // Possiblement rien
    PARAMETERS -> EXPRESSION PARAMETERS_NEXT
    PARAMETERS_NEXT -> , EXPRESSION PARAMETERS
    PARAMETERS_NEXT -> *
\end{verbatim}

Un point important pour clore cette section, est de noter que l'on rempli la
structure abstraite du programme au fur-et-à-mesure de l'analyse
("parameters\_add").

\subsection{Représentation abstraite du programme}

A chaque élément du langage nous avons associé une structure de donnée
correspondante, le tout formant directement notre arbre d'analyse, que l'on
rempli et vérifi au fur-et-à-mesure de l'analyse du programme.

Le schéma ci-dessous montre les relations existantes entre les différents
éléments du langage :

% TODO insertion du schéma
\begin{center}
\includegraphics[width=16cm]{structs-diagram.png}
\end{center}

\subsubsection{Représentation des types}

Le type est une structure de donnée récursive, dont la récursion dépends du
type de type :
\begin{verbatim}
/**
 * The type data structure. Have a look to type_type_t enumeration for more
 * details.
 */
struct type {
    type_type_t type;
    union {
        structure_t* structure_type;
        type_t* vector_type;
        type_t* optional_type;
        function_signature_t* signature;
    };
};
\end{verbatim}

Le champ type de cette structure est ou bien un type primitif (boolean,
integer, real...), ou bien un type plus complexe nécessitant des détails
supplémentaires.

Par exemple, lorsque le type est "TYPE\_TYPE\_VECTOR", le champs "vector\_type"
est affecté au type du vecteur.

\subsubsection{Représentation des expressions}

Les expressions forment un arbre binaire, dont les feuilles sont une valeur.

\begin{verbatim}
/**
 * Different kinds of expression nodes.
 */
typedef enum {
    EXPRESSION_TYPE_VALUE,
    EXPRESSION_TYPE_LAMBDA,

    EXPRESSION_TYPE_CMP_OP_EQUALS,
    EXPRESSION_TYPE_CMP_OP_DIFFERENT,
    EXPRESSION_TYPE_CMP_OP_LOWER_OR_EQUALS,
    EXPRESSION_TYPE_CMP_OP_GREATER_OR_EQUALS,
    EXPRESSION_TYPE_CMP_OP_LOWER,
    EXPRESSION_TYPE_CMP_OP_GREATER,

    EXPRESSION_TYPE_BOOL_OP_AND,
    EXPRESSION_TYPE_BOOL_OP_OR,
    EXPRESSION_TYPE_BOOL_OP_NOT,

    EXPRESSION_TYPE_ARITHMETIC_OP_PLUS,
    EXPRESSION_TYPE_ARITHMETIC_OP_MINUS,
    EXPRESSION_TYPE_ARITHMETIC_OP_MUL,
    EXPRESSION_TYPE_ARITHMETIC_OP_DIV,
    EXPRESSION_TYPE_ARITHMETIC_OP_MOD,

    EXPRESSION_TYPE_SIZE
} expression_type_t;

/**
 * Expression representation.
 *
 * An expression is a tree (number of childs of a given node depends of the
 * node's kind), where `type` is the kind of node.
 *
 * If a node is of kind `EXPRESSION_TYPE_VALUE`, it has its internal union
 * `value` set to a valid value.
 *
 * If it has a binary operator kind (all other kinds excepted
 * 'EXPRESSION_TYPE_BOOL_OP_NOT'), all `left` and `right` childs are set.
 * If it is of kind `EXPRESSION_TYPE_BOOL_OP_NOT`, only right child is set.
 */
struct expression {
    expression_type_t type;
    union {
        value_t value;
        function_t* lambda;
        type_t* empty_type;
        struct {
            struct expression *left, *right;
        };
    };
};
\end{verbatim}


\subsection{Vérification sémantique du programme}

\paragraph{}Classiquement durant la phase de vérification syntaxique, un 
compilateur créer un arbre de syntaxe abstraite, qui est ensuite utiliser afin 
de réaliser la vérification sémantique.

\paragraph{}La vérification sémantique ce fait ensuite via l'aide des tables 
suivantes :

\begin{itemize}
 \item la table des lexicographique ou table des symboles
 \item la table des déclarations
 \item la table des représentations
 \item la table des régions
\end{itemize}

\paragraph{}Plutôt que d'utiliser 4 tables de la sorte nous avons opté pour 
un \textbf{context d'analyse} qui est l'ensemble de :
\begin{itemize}
    \item Le programme courant que l'on analyse ;
    \item La fonction courante que l'on analyse.
\end{itemize}

\paragraph{}Le programme contient toutes les déclarations globales comme les 
structures, les variables et constantes globales, ainsi que les les functions 
et procédures.

\paragraph{}Les fonctions et procédures contiennent les parametres, les 
variables locales, ainsi que la liste d'instruction de la fonction et le type 
de retour pour les fonctions.

\paragraph{}Nous ne créons donc pas d'arbre syntaxique lors du parsing d'un 
programme ez, on remplit directement les différentes structures utiles. 

Nous utilisons toutefois une structure d'arbre pour stocker les expressions, 
ce qui nous permet de mettre en place une vérification en fonction de la 
priorité des opérateurs.

\paragraph{} La méthode de parsing étant différentes de celle générée par YACC, 
nous n'avons ps d'arbre implicite créer au moment du parcours du fichier. En 
effet, avec une grammaire LALR, on peut directement mettre en place la priorité 
des opérateurs au moment du parsing, il en résulte un arbre syntaxique 
implicite que nous n'avons qu'a explicité avec le structure souhaité.

Dans notre cas, cet arbre syntaxique implicite n'éxiste pas et les opérateurs 
on donc tous la même priorité, syntaxiquement parlant.

Dans le but de pouvoir effectuer différentes vérifications sur nos expressions, 
comme la résolution du type, nous avons dû traiter les expressions.

%% EXPLIQUER ALGO ??

\paragraph{}Le compilateur ezc ne fait donc qu'une passe pour effectuer les 
vérifications lexicale, syntaxique et sémantique d'un programme ez.

\paragraph{}Plusieurs vérifications sémantiques sont effectuées durant 
l'analyse du programme.

\begin{description}
 \item[Existance des symboles] Lorsque l'on rencontre un symbol on vérifie son 
existance. Pour une déclarations, le symbole ne doit pas être déjà existant. 
Lors de l'utilisation d'un symbole (variable ou bien appel de fonction), il doit
en revanche avoir été précédement déclaré. La recherche d'un symbole dans le
contexte ce fait dans l'ordre suivant :

 \begin{enumerate}
  \item Les variables locales de la fonction
  \item Les paramatres de la fonction
  \item Les variables globales du programme
  \item Les constantes du programme
  \item Les fonctions et procédures
 \end{enumerate}

 \item[Vérification des types] Lors du parsing d'une expression ou bien lors 
d'une affectation, le compilateur vérifie la compatibilité des types entre les
opérateurs. De même que pour la vérification de l'existance du symbol
on recherche le type du symbole rencontrer pour obtenir son type.

 \item[Vérification des types d'accès] Le compilateur vérifie également le type
d'acces des variables dans une expression, un paramatres de fonction
déclaré avec un type d'accès de sortie ne peut donc pas être utilisé en partie 
droite d'une affectation, alors qu'un parametres déclaré comme une entrée
seulement, ne peut pas être modifié.
 
 \item[Vérification des appels de fonction] Pour les appels de fonction, le 
compilateur s'assure que la signature de la fonction correspont bien avec 
l'appel qui en est fait, ausi bien au niveau du nombre d'argument que de leurs 
types.
\end{description}

\paragraph{}Il est à noter, que le fait de se passer de la table des symboles 
peut entrainer une perte de performance du compilateur, l'interet de cette 
table étant en effet d'associer les différents identifieurs à un label 
(typiquement grâce à une table de hachage) permettant de les 
retrouver en un temps constant.

Dans notre cas, nous éxecutons une simple recherche dans un tableau,
la tâche est donc alourdie. En revanche, là où classiquement après avoir 
trouver l'identifieur, il faut examiner la table des déclarations pour 
connaitre son type et sa portée, nous connaissont déjà ces informations grâce 
aux structures utilisées.

\subsection{Réécriture du programme}

Par soucis de clareté dans le code, la phase de réécriture en langage 
intermédiaire (dans notre cas C++) est elle réalisée lors d'un parcours du 
contexte généré lors de la première passe. On aurait cependant, très bien
pu imaginer n'effectuer qu'une seule et unique passe pour l'ensemble des 
étapes de compilations.


%% PARLER DE LA RÉÉCRITURE AVEC LES CLASSES CPP IMPLÉMENTÉE
